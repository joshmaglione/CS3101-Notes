\let\emphasized\emph
\let\strong\textbf
\renewcommand{\emph}[1]{\textbf{#1}}
\renewcommand{\textbf}[1]{\textcolor{Blue}{\strong{#1}}}
\usepackage{hyperref}
\usepackage{amsmath,amscd,amsthm,mathtools,qtree,tikz,tikz-cd,mathpazo}
\usepackage[T1]{fontenc}
\usepackage[svgnames]{xcolor}
\usepackage{minted}
\usepackage{graphicx}
\usepackage{url}
\usefonttheme{serif}
\usefonttheme{professionalfonts}
\hypersetup{
	colorlinks=true,
	linkcolor=blue,
	filecolor=blue,
	urlcolor=blue,
	citecolor=blue,
}

\newtheorem*{thm}{Theorem}
\newtheorem*{lem}{Lemma}
\newtheorem*{prop}{Proposition}
\newtheorem*{cor}{Corollary}
\theoremstyle{definition}
\newtheorem*{defn}{Definition}
\newtheorem*{exa}{Example}
\newtheorem*{nonexa}{Non-example}
\newtheorem*{exc}{Exercise}
\newtheorem*{rem}{Remark}
\newtheorem*{qu}{Question}
\renewcommand{\le}{\leqslant}
\renewcommand{\ge}{\geqslant}
\renewcommand{\leq}{\le}
\renewcommand{\geq}{\ge}
\DeclarePairedDelimiter{\abs}{\lvert}{\rvert}
\newcommand{\QED}{\hfill$\qed$}
\DeclareMathOperator{\Hom}{Hom}
\DeclareMathOperator{\End}{End}
\DeclareMathOperator{\Fun}{Fun}
\DeclareMathOperator{\Ker}{Ker}
\DeclareMathOperator{\Img}{Im}