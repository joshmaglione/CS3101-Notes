\documentclass[a4paper, 12pt]{article}

\usepackage{enumerate}
\usepackage{hyperref}
\hypersetup{
	colorlinks=true,
	linkcolor=blue,
	filecolor=blue,
	urlcolor=blue,
	citecolor=blue,
}
\usepackage{amsmath}
\usepackage{amsthm}
\usepackage{amssymb}
\usepackage[margin=3cm]{geometry}
\usepackage{mathpazo}
\usepackage{url}
\usepackage{subcaption}
\usepackage{tikz}
\usepackage{pgf}
\usepackage{longtable}
\usepackage{multirow}
\usepackage{graphicx}
\usepackage{pgfplots}
\usepackage{cleveref}
\usepackage{bbm}
\usepackage{wrapfig}
\usepackage{mathrsfs}
\usepackage{afterpage}

\numberwithin{equation}{section}
\numberwithin{figure}{section}

\newtheorem{thm}{Theorem}[section]
\newtheorem*{thm*}{Theorem}
\newtheorem*{con*}{Conjecture}
\newtheorem{lem}[thm]{Lemma}
\newtheorem{prop}[thm]{Proposition}
\newtheorem{cor}[thm]{Corollary}
\newtheorem{lemma}[thm]{Lemma}
\newtheorem{conj}[thm]{Conjecture}

\theoremstyle{definition}
\newtheorem{defn}[thm]{Definition}
\newtheorem{remark}[thm]{Remark}
\newtheorem{ex}[thm]{Example}
\newtheorem{quest}[thm]{Question}
\newtheorem{obs}[thm]{Observation}

\renewcommand{\leq}{\leqslant}
\renewcommand{\geq}{\geqslant}
\newcommand{\N}{\mathbb{N}}
\newcommand{\Z}{\mathbb{Z}}
\newcommand{\Q}{\mathbb{Q}}
\newcommand{\R}{\mathbb{R}}
\newcommand{\C}{\mathbb{C}}

\setcounter{tocdepth}{2}

\allowdisplaybreaks

\title{Software for Mathematical Scientists and Educators}
\author{Joshua Maglione}
\date{\today}

\begin{document}

\maketitle
\tableofcontents

\section*{Introduction}

Communicating mathematics and performing long computations are both vitally
important and challenging. Thankfully there is a wide selection of software to
make these tasks more manageable. In this module, we will explore software used
by everyday mathematicians and scientists. These include biologists, chemists,
computer scientists, data scientists, financial analysts, educators, engineers,
and physicists. 

The goal is to build a foundation by using some of the most ubiquitous software
in the field. This will help students throughout their career in and out of
university. We will cover four topics in this module:
\begin{enumerate} 
	\item Mathematical Typesetting and \LaTeX,
	\item Python and Jupyter Notebooks,
	\item Introduction to Programming,
	\item Symbolic Computation and SageMath.
\end{enumerate}


\section{Mathematical Typesetting and \LaTeX}

With advances in printing and typesetting, the question of how to produce
high-quality mathematical symbols and texts is challenging. Without going
through the history, we now have essentially two main styles of software to
write mathematical formulae, diagrams, and images:
\begin{enumerate}
	\item What-You-See-Is-What-You-Get (WYSIWYG) and
	\item typesetting software (or write-format-preview style).
\end{enumerate}

Software like Microsoft Word, Apple Pages, or LibreOffice Writer are WYSIWYG
editors because you see and edit the document as a final product (regardless of
whether or not it is the final product). This remove the user from having to
remember commands for the document layout, and it has a lower barrier of entry,
which is one of its strongest advantages. However, this is not the norm for
professionals using many mathematical symbols, and the primary disadvantage to
these kinds of software is their sluggish pace. 

One of the first typesetting software for mathematics is \TeX---if not
\textit{the} first. It was written by Donald Knuth\footnote{You can read Knuth's
original memo describing \TeX. He called the memo the ``Preliminary preliminary
description of TEX''~\cite{TeX-draft}.} in 1978~\cite{Knuth-Quanta}, and it is
the cornerstone of the more modern software \LaTeX\ written by Leslie Lamport in
1986~\cite{Lamport}. Although \TeX\ is still used today, \LaTeX\ is far more
popular. The primary disadvantage to these systems is the higher barrier to
entry, but the over-powering advantage is the speed with which one can produce
beautiful and high-quality mathematical symbols. Both \TeX\ and \LaTeX\ are free
and open-source software. 

As alluded to previously, these are typesetting software, so a user writes in a
markup language, usually in a \texttt{tex}-file; then the user compiles the
file, and a \texttt{pdf}-file is produced. There are other options for output,
but this will be sufficient for us. Another asset is that one can import third
party \LaTeX\ packages to perform more specialized tasks. We will become
familiar with basic \LaTeX\ formatting and use some packages to help construct
beautiful documents. 

For web-based mathematical symbols, MathJax~\cite{MathJax} is primarily used.
However, there is a new, much faster, alternative called KaTeX~\cite{KaTeX}.
Currently, KaTeX can only do a (proper) subset of what MathJax can do, but KaTeX
does enough for everything I have needed for my website. 

\subsection{How to get \LaTeX}



\subsection{Workflow and document structure}



\newpage

\bibliography{bibliography}
\bibliographystyle{plain}

\end{document}